

\newcommand{\matx}[1]{\mbox{\tt #1}} \newcommand{\vect}[1]{{\bf #1}}

\begin{itemize} \item n = number of objects aka independent motions aka subspaces \item P = number of $(x, y)$ point correspondences \item F = number
of frames \end{itemize}

Let us stack all the frames vertically so that we get a matrix W

$$W = MS^T$$

where M is motion matrix $\in \mathbb{R}^{2F \time 4}$ and where $S$ is shape matrix (point correspondences).

Additionally, we have the world coordinates (x,y,z) as

\begin{equation} X = \begin{bmatrix} x \\ y \\ z \\ 1 \\ \end{bmatrix} \end{equation}

\begin{equation} x = \begin{bmatrix} x \\ y \\ \end{bmatrix} \end{equation}

The affine camera matrix translating between a given world and image coordinate is written in terms of rotations and translations as A $\in
\mathbb{R}^{2 \times 4}$

\begin{equation} \vect A = \begin{bmatrix} R1 && R2 && R3 && T1 \\ R3 && R4 && R5 && T2 \\ \end{bmatrix} \end{equation}

 \begin{figure} \centering
     % \includegraphics[width=0.3\textwidth]{./pca.png}
     \caption{If we imagine the blobs as clouds of points in 3D (so imagine them going at some angle through space), the  problem with picking a
 single subspace is that the data lies more naturalies in two subspaces, e.g. the two lines going through the gray and dark gray clusters.}
 \end{figure}

\subsubsection{Representing motion subspaces with polynomials}

 The gist of this section is that, if all n subspaces (each representing one object) have dimension 4 in $\mathbb{R}^5$, then a single linear
 polynomial of degree 5 is enough to represent them. To see this, consider that a plane may be represented by a linear polynomial of three variables,
 $ax+by+cz = 0$. Then, we can fit a polynomial defined as the product of the n planes, which will have at most degree n and have three variables. We
 can also describe lower dimension subsubspaces as a the "common zero set" of multiple hyperplanes. For instance, a line can be described as the
 intersection of two hyperplanes. 

 \begin{figure} \centering
     % \includegraphics[width=0.3\textwidth]{./polysubspace.png}
     \caption{Here the intersection, or combined zero-set, of two 3D planes are used to define a line (2D subspace) in $\mathbb{R}^3$.} \end{figure}

 \textit{Aside: an algebraic variety is roughly like a "shape" that fits the points. Here we are saying that if this polynomial "shape" fits all the
 datapoints in n subspaces, then we should be able to factor the algebraic variety nto n polynomials.  Co-dimension means the number of dimensions not
 occupied by a subspace, e.g. a subspace of dimension 4 in 5 dimensional space has a co-dimension of 1. }
 % TODO: zero set
