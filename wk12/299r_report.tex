\documentclass[a4paper]{article}

%% Language and font encodings
\usepackage[english]{babel}
\usepackage[utf8x]{inputenc}
\usepackage[T1]{fontenc}

%% Sets page size and margins
\usepackage[a4paper,top=3cm,bottom=2cm,left=3cm,right=3cm,marginparwidth=1.75cm]{geometry}

%% Useful packages
\usepackage{amsmath}
\usepackage{amssymb}
\usepackage{graphicx}
\usepackage[colorinlistoftodos]{todonotes}
\usepackage[colorlinks=true, allcolors=blue]{hyperref}
\usepackage{url}
\usepackage{listings}
\PassOptionsToPackage{hyphens}{url} 

\setlength{\parskip}{\baselineskip}%
\setlength{\parindent}{0pt}%

\title{Sampling Based Motion Planning Algorithms}
\author{299r Fall 2017 by Nao Ouyang}

\begin{document}
\maketitle

\begin{abstract} 
This paper covers the work I did for my research rotation with Professor Lucas Janson in Fall of 2017. The rotation focused around sampling-based
motion planning algorithms which are frequently used in robotics (my area of interest).  The rotation can be roughly divided into three main
components: 1. Understanding the algorithms 2. Understanding the proofs of the algorithmic properties 3. Computer simulations of the algorithms
The specific algorithms covered include: 1. RRT and it's optimal variant, RRT* 2. PRM and it's simplified variant, sPRM 3. FMT* Finally, I'll write a section reflecting on my experiences this semester.

\end{abstract}

\section{Introduction}


\section{Algorithms}

There are broadly two different approaches to sampling-based motion planning. In all cases, the sampling part is in reference to the fact that, as




\section{Methods}


\subsection{Problem Formulation}

\newcommand{\matx}[1]{\mbox{\tt #1}} \newcommand{\vect}[1]{{\bf #1}}

\begin{itemize} \item n = number of objects aka independent motions aka subspaces \item P = number of $(x, y)$ point correspondences \item F = number
of frames \end{itemize}

Let us stack all the frames vertically so that we get a matrix W

$$W = MS^T$$

where M is motion matrix $\in \mathbb{R}^{2F \time 4}$ and where $S$ is shape matrix (point correspondences).

Additionally, we have the world coordinates (x,y,z) as

\begin{equation} X = \begin{bmatrix} x \\ y \\ z \\ 1 \\ \end{bmatrix} \end{equation}


\begin{equation} x = \begin{bmatrix} x \\ y \\ \end{bmatrix} \end{equation}



The affine camera matrix translating between a given world and image coordinate is written in terms of rotations and translations as A $\in
\mathbb{R}^{2 \times 4}$

\begin{equation} \vect A = \begin{bmatrix} R1 && R2 && R3 && T1 \\ R3 && R4 && R5 && T2 \\ \end{bmatrix} \end{equation}

 \begin{figure} \centering
     % \includegraphics[width=0.3\textwidth]{./pca.png}
     \caption{If we imagine the blobs as clouds of points in 3D (so imagine them going at some angle through space), the  problem with picking a
 single subspace is that the data lies more naturalies in two subspaces, e.g. the two lines going through the gray and dark gray clusters.}
 \end{figure}

\subsubsection{Representing motion subspaces with polynomials}

 The gist of this section is that, if all n subspaces (each representing one object) have dimension 4 in $\mathbb{R}^5$, then a single linear
 polynomial of degree 5 is enough to represent them. To see this, consider that a plane may be represented by a linear polynomial of three variables,
 $ax+by+cz = 0$. Then, we can fit a polynomial defined as the product of the n planes, which will have at most degree n and have three variables. We
 can also describe lower dimension subsubspaces as a the "common zero set" of multiple hyperplanes. For instance, a line can be described as the
 intersection of two hyperplanes. 


 \begin{figure} \centering
     % \includegraphics[width=0.3\textwidth]{./polysubspace.png}
     \caption{Here the intersection, or combined zero-set, of two 3D planes are used to define a line (2D subspace) in $\mathbb{R}^3$.} \end{figure}

 \textit{Aside: an algebraic variety is roughly like a "shape" that fits the points. Here we are saying that if this polynomial "shape" fits all the
 datapoints in n subspaces, then we should be able to factor the algebraic variety nto n polynomials.  Co-dimension means the number of dimensions not
 occupied by a subspace, e.g. a subspace of dimension 4 in 5 dimensional space has a co-dimension of 1. }
 % TODO: zero set
\section{Results}

I used the dataset Hopkins 155 provided by JHU. It is available at http://www.vision.jhu.edu/data/.

I created a MATLAB file which plots the datapoints in the .mat files against the video. This file may be found as . In order to run

My code appeared to have an error rate of 35\% for the 'arm' dataset in Pt 6 of Hokins 155. This was unfortunately much larger than the reported 5\%
error for most of the datasets.

I did not get around to plotting the points by color onto the video to debug what was going on.

%%%%%%%%%%%%%%%%%%%%%%%%%%%%%%%

\section{Conclusion}

Powerfactorization was not implemented, and the code appears to still have issues. To Debug, I should assign colors to my labels and check what might
be causes of the issue.  Future work includes playing around with RANSAC with outliers.

%%%%%%%%%%%%%%%%%%%%%%%%%%%%%%%%

\section{References}

 
R. Vidal and R. Hartley, "Motion segmentation with missing data using PowerFactorization and GPCA," Proceedings of the 2004 IEEE Computer Society
Conference on Computer Vision and Pattern Recognition, 2004. CVPR 2004., doi: 10.1109/CVPR.2004.1315180.
\href{http://vision.jhu.edu/assets/VidalCVPR04.pdf}{Paper link}

\bigskip

\noindent Hartley R.I. and Tron, R., \& Vidal, R. (2007). Multiframe Motion Segmentation with Missing Data Using PowerFactorization and GPCA.
International Journal of Computer Vision, 79, 85-105. doi: 10.1007/s11263-007-0099-z.
\href{https://pdfs.semanticscholar.org/322b/3feffe79585c112c6a0d86c05909634ac7c5.pdf}{Paper link}

%%%%%%%%%%%%%%%%%%%%%%%%%%%%%%%%%
%http://cs229.stanford.edu/notes/cs229-notes10.pdf (PCA)


%PCA (Andrew Ng) https://www.youtube.com/watch?v=T-B8muDvzu0&index=83&list=PLLssT5z_DsK-h9vYZkQkYNWcItqhlRJLN


%https://link-springer-com.ezp-prod1.hul.harvard.edu/book/10.1007%2F978-0-387-87811-9#about
\end{document}

